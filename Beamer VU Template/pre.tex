% IMPORTANT: PLEASE USE XeLaTeX FOR TYPESETTING
% \documentclass{sintefbeamer}

% packages, font, color, and newcommands
% \usepackage{amsfonts, amsmath, oldgerm, lmodern}
% \usepackage{xeCJK}
% \usefonttheme{serif}

% meta-data
% \title{Beamer}
% \subtitle{subtitle}
% \author{\href{j.busink@vu.nl}{}}
% \date{\today}
% \titlebackground{images/VuBG2.png}
% \setbeamertemplate{sidebar right}{}
% \setbeamertemplate{footline}{%
% \hfill\usebeamertemplate***{navigation symbols}
% \hspace{1cm}\insertframenumber{}/\inserttotalframenumber}
% document body
\begin{document}

% \maketitle

% \begin{frame}

% \hrefcol{mailto:j.busink@svu.nl}{Joris Busink} 

% \end{frame}

% \section{Section 1}

% \begin{frame}{Beamer slides}{\thesection \, \secname}

% \begin{itemize}
% \item We assume you can use \LaTeX; if you cannot,
% \hrefcol{http://en.wikibooks.org/wiki/LaTeX/}{you can learn it here}
% \item Beamer is one of the most popular and powerful document
% classes for presentations in \LaTeX
% \end{itemize}
% \end{frame}


% \section{Section 2}

% \begin{frame}[fragile]{Selecting the Class}
% After the last update to the graphic profile, the \texttt{sintef} theme for
% Beamer has been updated into a full-fledged class.
% To start working with \texttt{sintefbeamer}, start a \LaTeX\ document with the
% preamble:
% \begin{block}{Minimum SINTEF Beamer Document}
% \begin{lstlisting}[language=TeX]
% \documentclass{sintefbeamer}
% \begin{document}
% \begin{frame}{Hello, world!}
% \end {frame}
% \end{document}
% \end{lstlisting}
% \end{block}
% \end{frame}

% \begin{frame}[fragile]{Simple Slide}
% \framesubtitle{Subtitle}
% \begin{itemize}[<+->]
% \item Bullet 1
% \item Bullet 2
% \item Bullet 3
% \end{itemize}
% \begin{block}{Code for a Page with an Itemised List}<+->
% \begin{lstlisting}[language=TeX]
% \begin{frame}
%   \frametitle{Writing a Simple Slide}
%   \framesubtitle{It's really easy!}
%   \begin{itemize}[<+->]
%     \item A typical slide has bulleted lists
%     \item These can be uncovered in sequence
%   \end{itemize}
% \end{frame}\end{lstlisting}
% \end{block}
% \end{frame}

% \begin{frame}[fragile]{Look}
% \begin{itemize}
% \item To change the colour of the title dash, give one of the class options
%       \texttt{cyandash} (default), \texttt{greendash}, \texttt{magentadash},
%       \texttt{yellowdash}, or \texttt{nodash}.
% \item To change between the light and dark themes, give the class options
%       \texttt{light} (default) or \texttt{dark}. It is not possible to switch
%       theme for one slide because of the design of Beamer---and it's probably a
%       good thing.
% \item To insert a final slide, use \verb|\backmatter|.
% \item The aspect ratio defaults to 16:9, but you can change it to 4:3 for old
%       projectors by passing the class option \texttt{aspectratio=43}; any other
%       values accepted by Beamer are also possible.
% \end{itemize}
% \end{frame}

% \section{Section 3}

% \begin{frame}
% \frametitle{Frame Title}
% \begin{itemize}
% \item Some text
% \end{itemize}
% \end{frame}

% \backmatter

\end{document}
